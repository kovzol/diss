\documentclass[a4paper]{article}
\usepackage{gclc_proof}


\begin{document}

\title{GCLC Prover Output for conjecture ``thm-0110-Parallelogram2'' }

\author{Groebner bases method used}

\maketitle





\section{Construction and prover internal state}





\subsection*{Construction commands:}

\begin{itemize}

\item 

Point $A$ 
\item 

Point $B$ 
\item 

Point $C$ 
\item 

Line $ab$: $A$ $B$ 
\item 

Line $bc$: $B$ $C$ 
\item 

Parallel, $p$: $A$ $bc$ 
\item 

Parallel, $q$: $C$ $ab$ 
\item 

Intersection of lines, $D$: $p$ $q$ 
\item 

Line $ac$: $A$ $C$ 
\item 

Line $bd$: $B$ $D$ 
\item 

Intersection of lines, $E$: $ac$ $bd$ 
\end{itemize}





\subsection*{Coordinates assigned to the points:}

\begin{itemize}

\item $A = (0, 0)$

\item $B = (u_{1}, 0)$

\item $C = (u_{2}, u_{3})$

\item $D = (x_{2}, u_{3})$

\item $E = (x_{4}, x_{3})$

\end{itemize}





\subsection*{Conjecture(s):}

\begin{enumerate}

\item Given conjecture

\begin{itemize}

\item \textbf{GCLC code:}


\begin{verbatim}
 same_length A E E C 
\end{verbatim}

\end{itemize}

\begin{itemize}

\item \textbf{Expression:}



$AE\cong EC $\end{itemize}

\end{enumerate}

\hspace*{4.1mm}





\section{Resolving constructed lines}

\begin{itemize}

\item $ab \ni A, B$ ; line is horizontal (i.e., $y(A) = y(B)$)\item $bc \ni B, C$ \item $p \ni A, D$ \item $q \ni C, D$ ; line is horizontal (i.e., $y(C) = y(D)$)\item $ac \ni A, C, E$ \item $bd \ni B, D, E$ \end{itemize}





\section{Creating polynomials from hypotheses}

\begin{itemize}

\item 

Point $A$ 


no condition\item 

Point $B$ 


no condition\item 

Point $C$ 


no condition\item 

Line $ab$: $A$ $B$ 
\begin{itemize}

\item 

point $A$ is on the line ($A$, $B$)


no condition\item 

point $B$ is on the line ($A$, $B$)


no condition\end{itemize}

\item 

Line $bc$: $B$ $C$ 
\begin{itemize}

\item 

point $B$ is on the line ($B$, $C$)


no condition\item 

point $C$ is on the line ($B$, $C$)


no condition\end{itemize}

\item 

Parallel, $p$: $A$ $bc$ 
\begin{itemize}

\item Line ($A$, $D$) parallel with line ($B$, $C$)

$$
p_{1}  =  u_{3}x_{2}+(-u_{3}u_{2}+u_{3}u_{1})
$$
\end{itemize}

\item 

Parallel, $q$: $C$ $ab$ 
\begin{itemize}

\item Line ($C$, $D$) parallel with line ($A$, $B$)

 --- true by the construction

\end{itemize}

\item 

Intersection of lines, $D$: $p$ $q$ 
\begin{itemize}

\item 

point $D$ is on the line ($A$, $D$)


no condition\item 

point $D$ is on the line ($C$, $D$)


no condition\end{itemize}

\item 

Line $ac$: $A$ $C$ 
\begin{itemize}

\item 

point $A$ is on the line ($A$, $C$)


no condition\item 

point $C$ is on the line ($A$, $C$)


no condition\end{itemize}

\item 

Line $bd$: $B$ $D$ 
\begin{itemize}

\item 

point $B$ is on the line ($B$, $D$)


no condition\item 

point $D$ is on the line ($B$, $D$)


no condition\end{itemize}

\item 

Intersection of lines, $E$: $ac$ $bd$ 
\begin{itemize}

\item 

point $E$ is on the line ($A$, $C$)
$$
p_{2}  =  -u_{3}x_{4}+u_{2}x_{3}
$$
\item 

point $E$ is on the line ($B$, $D$)
$$
p_{3}  =  -u_{3}x_{4}+x_{3}x_{2}-u_{1}x_{3}+u_{3}u_{1}
$$
\end{itemize}

\end{itemize}





\section{Creating polynomial from the conjecture}

\begin{itemize}

\item Processing given conjecture(s).

\item 
 Segment [$A$, $E$] equal size as segment [$E$, $C$]

$$
p_{4}  =  2u_{2}x_{4}+2u_{3}x_{3}+(-u_{3}^{2}-u_{2}^{2})
$$
\end{itemize}

\begin{description}

\item [Conjecture 1:] $$
p_{5}  =  2u_{2}x_{4}+2u_{3}x_{3}+(-u_{3}^{2}-u_{2}^{2})
$$
\end{description}





\section{Invoking the theorem prover}

The used proving method is Buchberger's method.

Input polynomial system is:

\begin{eqnarray*}
p_{0} &=& u_{3}x_{2}+(-u_{3}u_{2}+u_{3}u_{1})\\
p_{1} &=& -u_{3}x_{4}+u_{2}x_{3}\\
p_{2} &=& -u_{3}x_{4}+x_{3}x_{2}-u_{1}x_{3}+u_{3}u_{1}\\
\end{eqnarray*}




\subsection{Iteration 1}

Current set is $S_{1} = $\begin{eqnarray*}
p_{0} &=& u_{3}x_{2}+(-u_{3}u_{2}+u_{3}u_{1})\\
p_{1} &=& -u_{3}x_{4}+u_{2}x_{3}\\
p_{2} &=& -u_{3}x_{4}+x_{3}x_{2}-u_{1}x_{3}+u_{3}u_{1}\\
\end{eqnarray*}
\begin{enumerate}

\item Creating S-polynomial from the pair $(p_{0}, p_{1}$).

Skipping pair $p_{0}$ and $p_{1}$ because gcd of their leading monoms is zero.\item Creating S-polynomial from the pair $(p_{0}, p_{2}$).

Skipping pair $p_{0}$ and $p_{2}$ because gcd of their leading monoms is zero.\item Creating S-polynomial from the pair $(p_{1}, p_{2}$).

Forming S-pol of $p_{1}$ and $p_{2}$:$$
p_{12}  =  u_{3}x_{3}x_{2}+(-u_{3}u_{2}-u_{3}u_{1})x_{3}+u_{3}^{2}u_{1}
$$
S-pol added.\end{enumerate}





\subsection{Iteration 2}

Current set is $S_{2} = $\begin{eqnarray*}
p_{0} &=& u_{3}x_{2}+(-u_{3}u_{2}+u_{3}u_{1})\\
p_{1} &=& -u_{3}x_{4}+u_{2}x_{3}\\
p_{2} &=& -u_{3}x_{4}+x_{3}x_{2}-u_{1}x_{3}+u_{3}u_{1}\\
p_{3} &=& -2u_{3}^{2}u_{1}x_{3}+u_{3}^{3}u_{1}\\
\end{eqnarray*}
\begin{enumerate}

\item Creating S-polynomial from the pair $(p_{0}, p_{3}$).

Skipping pair $p_{0}$ and $p_{3}$ because gcd of their leading monoms is zero.\item Creating S-polynomial from the pair $(p_{1}, p_{3}$).

Skipping pair $p_{1}$ and $p_{3}$ because gcd of their leading monoms is zero.\item Creating S-polynomial from the pair $(p_{2}, p_{3}$).

Skipping pair $p_{2}$ and $p_{3}$ because gcd of their leading monoms is zero.\end{enumerate}





\subsection{Groebner Basis}

Groebner basis has 4 polynomials:\begin{eqnarray*}
p_{0} &=& u_{3}x_{2}+(-u_{3}u_{2}+u_{3}u_{1})\\
p_{1} &=& -u_{3}x_{4}+u_{2}x_{3}\\
p_{2} &=& -u_{3}x_{4}+x_{3}x_{2}-u_{1}x_{3}+u_{3}u_{1}\\
p_{3} &=& -2u_{3}^{2}u_{1}x_{3}+u_{3}^{3}u_{1}\\
\end{eqnarray*}
Groebner basis succesfully computed.





\section{Reducing Polynomial Conjecture}

Reducing with polynomial $p_{1}$, the result is:

$$
p_{21}  =  (-2u_{3}^{2}-2u_{2}^{2})x_{3}+(u_{3}^{3}+u_{3}u_{2}^{2})
$$
Reducing with polynomial $p_{3}$, the result is:

$$
p_{22}  =  0
$$
Conclusion is reduced to zero.





\section{Prover report}

\begin{description}

\item [Status:]  The conjecture has been proved.

\item [Space Complexity:]  The biggest polynomial obtained during proof process contained 4 terms.

\item [Time Complexity:]  Time spent by the prover is 0.001 seconds.There are no ndg conditions.

\end{description}

\end{document}
